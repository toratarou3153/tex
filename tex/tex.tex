\documentclass{jarticle}
\usepackage{a4wide}
\usepackage{amsmath}%数学記号
\usepackage{amssymb}%数学記号
\usepackage{epsfig}%図
\usepackage{latexsym}
\usepackage{supertabular}
\usepackage{graphicx}
\usepackage{color}
\usepackage{ascmac}
\usepackage{multicol}
\usepackage{ascmac}
\pagestyle{plain}

\newtheorem{theorem}{定理}[section]
\newtheorem{lemma}[theorem]{補題}
\newtheorem{proposition}[theorem]{命題}
\newtheorem{conjecture}[theorem]{予想}
\newtheorem{corollary}[theorem]{系}
\newtheorem{definition}[theorem]{定義}
\newtheorem{example}[theorem]{例}
\newtheorem{exercise}[theorem]{例題}
\newtheorem{problem}[theorem]{問}
\newtheorem{algorithm}[theorem]{アルゴリズム}
\newtheorem{remark}[theorem]{注意}

\def\qed{{\hfill$\square$}}
\def\proof{{\vspace{-0.3cm}f 証明: \,}}
\def\solution{{\vspace{-0.3cm}f 解: \,}}
\def\N{{\Bbb N}}
\def\Z{{\Bbb Z}}
\def\Q{{\Bbb Q}}
\def\R{{\Bbb R}}
\def\C{{\Bbb C}}
\def\F{{\Bbb F}}
\def\D{{\mathcal D}}
\def\mod{{\mathrm{mod\,\,}}}
\def\GL{{\mathrm{GL}}}
\def\GF{{\mathrm{GF}}}
\def\H{{\mathcal{H}}}

\setlength{\textwidth}{170mm}
\setlength{\textheight}{240mm}
 \setlength{\oddsidemargin}{-5mm}
 \setlength{\evensidemargin}{-5mm}
\setlength{\topmargin}{-10mm}
\setlength{\headheight}{0mm}
\setlength{\headsep}{10mm}

\title{項目反応理論}
\date{}
\begin{document}
\maketitle
\section{項目分析}
\subsection{信頼性}
テストで測られる学力や性格、技能は真の得点呼ばれ、無視できない誤差を伴ってしか測定できないものである。そこで、テスト得点$\displaystyle x$を、真の得点$\displaystyle t$と誤差$\displaystyle e$の和
\begin{eqnarray}
\displaystyle x = t+ e
\end{eqnarray}
として表現する。

ここで誤差$\displaystyle e$と$\displaystyle t$に関する期待値を
  \begin{itembox}[l]{仮定1:  $\displaystyle E[e] = 0$}
    $\displaystyle +$の誤差の人、$\displaystyle -$の誤差の人がそれぞれ存在するが平均すれば$\displaystyle 0$になるだろうという仮定。
  \end{itembox}
\begin{itembox}[l]{仮定2:   $\displaystyle E[et] = 0$}
    真の点数が高いほど誤差が大きくなるなどのような関連性はなく、誤差はに無相関に散らばるという仮定。
  \end{itembox}

これら$2$の仮定の下で、テスト得点$\displaystyle x$の母平均は
\begin{eqnarray}
  \displaystyle \mu_{x}=E[x]=E[e+t]=E[t]=\mu_{t}
\end{eqnarray}
となり真の得点の母平均に一致する。また、テスト得点$\displaystyle x$分散は、
\begin{eqnarray}
  \displaystyle {\sigma_{x}}^2={\sigma_{e}}^2 + {\sigma_{t}}^2
\end{eqnarray}
のように真の得点の分散と、誤差の分散の単純な和に分解される。このとき、真の得点の分散とテスト得点の分散の比を信頼性係数という。
\begin{itembox}[l]{テスト得点$\displaystyle x$の信頼性係数}
   \begin{eqnarray}
    \label{07}
    \displaystyle \rho=\frac{ {\sigma_{t}}^2}{{\sigma_{e}}^2 + {\sigma_{t}}^2}=1-\frac{{\sigma_{e}}^2}{{\sigma_{x}}^2}=\frac{{\sigma_{t}}^2}{{\sigma_{x}}^2}
   \end{eqnarray}
\end{itembox}

\subsubsection{タウ等価測定}
テスト得点$\displaystyle x$だけでなく、項目得点$\displaystyle x_{i}$もまた、
\begin{eqnarray}
  \displaystyle x_{i} = t_{i}+e_{i}
\end{eqnarray}
というように古典的テストモデルに従っているものとし、項目$i$と項目$j$に関して、\\
\\
\begin{enumerate}
  \item  $E[e_{i}] = 0$
  \item  $E[t_{i}e_{i}] = 0$ ($i$=$j$のときを含む)
  \item  $E[e_{i}e_{j}] = 0$ ($i$ $\neq$ $j$の場合のみ)
\end{enumerate}
を仮定する。この仮定を満たしているとき、項目を互いに同族測定という。

また、
\begin{eqnarray}
  \displaystyle t_{1}-\mu_{t_{1}}=\cdots=t_{i}-\mu_{t_{i}}=\cdots=t_{n}-\mu_{t_{n}}=t^{\ast}
\end{eqnarray}
が、成り立つとき項目は互いにタウ等価測定であるという。言い換えると、タウ等価測定とは、
\begin{eqnarray}
  \displaystyle t-\mu_{t}=nt^{\ast}
\end{eqnarray}
であり、すべての項目は同じ真の得点を測っているという仮定である。したがって、タウ等価測定のもとでは
\begin{eqnarray}
  \displaystyle {\sigma_{t_{1}}}^2=\cdots={\sigma_{t_{i}}}^2=\cdots={\sigma_{t_{n}}}^2={\sigma_{t^\ast}}^2
\end{eqnarray}
のように、真の得点の分散は添え字によらず一定値$\displaystyle  {\sigma_{t^\ast}}^2$となる。さらに任意の$2$つの項目得点の共分散も
\begin{eqnarray}
  \displaystyle C[x_{i},x_{j}]={\sigma_{t^\ast}}^2
\end{eqnarray}
となる。
\subsubsection{クロンバックの$\displaystyle \alpha$係数}
タウ等価測定が成り立てば、信頼係数は
\begin{eqnarray}
  \label{02}
  \displaystyle \rho=\frac{V[t]}{{\sigma_{x}}^2}=\frac{V[nt^{\ast}]}{{\sigma_{x}}^2}
\end{eqnarray}
項目$i$と項目$j$の標本共分散$s_{ij}$はすべて$\displaystyle  {\sigma_{t^\ast}}^2$の推定量に使用できる。今回は精度を上げるために、$n(n-1)$個の標本共分散の平均
\begin{eqnarray}
  \label{01}
  \displaystyle \hat{\sigma}_{t^\ast}^2=\frac{1}{n(n-1)}\sum_{i=1}^{n} \sum_{i\neq j,j-1}^{n} s_{ij}
\end{eqnarray}
を推定量として利用する。式(\ref{01})を式(\ref{02})に代入して${s_{x}}^2$を${\sigma_{x}}^2$の推定量に利用すると
\begin{eqnarray}
  \displaystyle \alpha=\frac{n}{n-1} \frac{\sum_{i=1}^{n} \sum_{i\neq j,j-1}^{n} s_{ij}}{{s_{x}}^2}=\frac{n}{n-1} \left(\frac{1-\sum_{i=1}^{n} {s_{i}}^2} {{s_{x}}^2}\right)
\end{eqnarray}
のようにテスト得点$x$の信頼性係数の推定量を構成することができる。またこれをクロンバックの$\alpha$係数という。
\subsubsection{マクドナルドの$\displaystyle \Omega$係数}
クロンバックの$\alpha$係数はタウ等価測定を仮定して導出した。しかし、実際のテストにおいては$t$に対して等価であるとは言い難い場面が多い。そこで、同族測定の仮定だけで信頼性を推定してみる。最初の仮定を確認すると
\begin{screen}
  \begin{enumerate}
    \item  $E[e_{i}] = 0$
    \item  $E[t_{i}e_{i}] = 0$ ($i$=$j$のときを含む)
    \item  $E[e_{i}e_{j}] = 0$ ($i$ $\neq$ $j$の場合のみ)
  \end{enumerate}
\end{screen}
これより
\begin{eqnarray}
  \label{06}
  \displaystyle {\sigma_{e}}^2=\sum_{i=1}^{n} {\sigma_{ei}}^2
\end{eqnarray}
を得る。式(\ref{06})を式(\ref{07})に代入した式
\begin{eqnarray}
  \label{08}
\displaystyle \Omega=1-\frac{\sum_{i=1}^{n} {\sigma_{ei}}^2}{{\sigma_{x}}^2}
\end{eqnarray}
をマクドナルドの$\Omega$係数という。

\end{document}
