\documentclass[12pt]{jarticle}
\usepackage{a4wide}
\usepackage{amsmath}%数学記号
\usepackage{amssymb}%数学記号
\usepackage{epsfig}%図
\usepackage{latexsym}
\usepackage{supertabular}
\usepackage{graphicx}
\usepackage{color}
\usepackage{ascmac}
\usepackage{multicol}
\usepackage{ascmac}
\usepackage{systeme}
\usepackage{amsmath,cases}
\pagestyle{plain}

\newtheorem{theorem}{定理}[section]
\newtheorem{lemma}[theorem]{補題}
\newtheorem{proposition}[theorem]{命題}
\newtheorem{conjecture}[theorem]{予想}
\newtheorem{corollary}[theorem]{系}
\newtheorem{definition}[theorem]{定義}
\newtheorem{example}[theorem]{例}
\newtheorem{exercise}[theorem]{例題}
\newtheorem{problem}[theorem]{問}
\newtheorem{algorithm}[theorem]{アルゴリズム}
\newtheorem{remark}[theorem]{注意}

\def\qed{{\hfill$\square$}}
\def\proof{{\vspace{-0.3cm}f 証明: \,}}
\def\solution{{\vspace{-0.3cm}f 解: \,}}
\def\N{{\Bbb N}}
\def\Z{{\Bbb Z}}
\def\Q{{\Bbb Q}}
\def\R{{\Bbb R}}
\def\C{{\Bbb C}}
\def\F{{\Bbb F}}
\def\D{{\mathcal D}}
\def\mod{{\mathrm{mod\,\,}}}
\def\GL{{\mathrm{GL}}}
\def\GF{{\mathrm{GF}}}
\def\H{{\mathcal{H}}}

\setlength{\textwidth}{170mm}
\setlength{\textheight}{240mm}
 \setlength{\oddsidemargin}{-5mm}
 \setlength{\evensidemargin}{-5mm}
\setlength{\topmargin}{-10mm}
\setlength{\headheight}{0mm}
\setlength{\headsep}{10mm}

\title{項目反応理論}
\begin{document}
\maketitle
\section{ロジスティックモデル}
\subsection{オッズ・ロジットと1母数・2母数モデル}
あるサッカーチーム$i$とサッカーチーム$j$の試合の賭けを募集したとする。この時に発表されるのが、オッズと呼ばれ払戻金倍率の逆数のことである。例えば、オッズが$0.25$なら、その逆数の$4$倍の金額が払い戻される。$1000$円を賭けていたら、その元金にプラスして$4000$円を手に入れることができる。
オッズは
\begin{eqnarray}
  \label{00}
  \displaystyle odds = \frac{p_j(\theta_i)}{1 - p_j(\theta_i)}
\end{eqnarray}
と表現される。オッズが$0.25$のとき、式は$\displaystyle \frac{0.2}{1 - 0.2} =\frac{0.2}{0.8}$となっている。これは、チーム$i$が$0.2$の確率で勝つと評価されている。

逆にこの式を$p_j(\theta_i)$について解くと、
\begin{eqnarray}
  \label{01}
  \displaystyle  p_j(\theta_i) = \frac{odds_i}{1 + ddds_i}
\end{eqnarray}
となる。この式に合わせて、$1$母数モデルと$2$母数モデルの$ICC$を書き換えると、
\begin{eqnarray}
  \label{02}
  \displaystyle  p_j(\theta_i) &=& \frac{\exp(Da(\theta_i - b_j))}{1 + \exp(Da(\theta_i - b_j))} \\
  p_j(\theta_i) &=& \frac{\exp(Da_j(\theta_i - b_j))}{1 + \exp(Da_j(\theta_i - b_j))}
\end{eqnarray}
となり、$\exp(Da_j(\theta_i - b_j))$は被験者$i$が項目$j$に正答するオッズである。
\begin{itembox}[l]{オッズについての考察}
  自分のチームが弱い場合と強い場合について考える。
  \begin{itemize}
  \item $1$勝$8$敗のとき
  \begin{eqnarray}
    \label{03}
    \displaystyle \frac{\frac{1}{9}}{1 - \frac{1}{9}} = 0.125
  \end{eqnarray}
  \item $1$勝$16$敗のとき
  \begin{eqnarray}
    \label{04}
    \displaystyle \frac{\frac{1}{17}}{1 - \frac{1}{17}} = 0.062
  \end{eqnarray}
  \item $8$勝$1$敗のとき
  \begin{eqnarray}
    \label{05}
    \displaystyle \frac{\frac{8}{9}}{1 - \frac{8}{9}} = 8
  \end{eqnarray}
  \item $16$勝$1$敗のとき
  \begin{eqnarray}
    \label{06}
    \displaystyle \frac{\frac{16}{17}}{1 - \frac{16}{17}} = 16
  \end{eqnarray}
\end{itemize}
負けが重なっても、式の性質上値が$0$を下回ることがないが、価値が重なるときの値は、上にどこまでも伸びる。尺度の違いからうまく判定できないところに注意する必要がある。
\end{itembox}

上記の尺度の問題を解決する方法として、オッズの対数をとる操作を行う。式にすると
\begin{eqnarray}
  \label{07}
  \displaystyle logit_i = log(odds_i) = log(\frac{p_j(\theta_i)}{1 - p_j(\theta_i)})
\end{eqnarray}
と表され、これをロジットという。上記を例に考えると、
  \begin{itemize}
  \item $1$勝$8$敗のとき
  \begin{eqnarray}
    \label{08}
    \displaystyle log \left(\frac{\frac{1}{9}}{1 - \frac{1}{9}}\right) \fallingdotseq -0.903
  \end{eqnarray}
  \item $1$勝$16$敗のとき
  \begin{eqnarray}
    \label{09}
    \displaystyle log \left(\frac{\frac{1}{17}}{1 - \frac{1}{17}}\right) \fallingdotseq -1.207
  \end{eqnarray}
  \item $8$勝$1$敗のとき
  \begin{eqnarray}
    \label{10}
    \displaystyle log \left(\frac{\frac{8}{9}}{1 - \frac{8}{9}}\right) \fallingdotseq 0.903
  \end{eqnarray}
  \item $16$勝$1$敗のとき
  \begin{eqnarray}
    \label{11}
    \displaystyle log \left(\frac{\frac{16}{17}}{1 - \frac{16}{17}}\right) \fallingdotseq 1.204
  \end{eqnarray}
\end{itemize}
このように尺度をそろえることができる。







\end{document}
