\documentclass[12pt]{jarticle}
\usepackage{a4wide}
\usepackage{amsmath}%数学記号
\usepackage{amssymb}%数学記号
\usepackage{epsfig}%図
\usepackage{latexsym}
\usepackage{supertabular}
\usepackage{graphicx}
\usepackage{color}
\usepackage{ascmac}
\usepackage{multicol}
\usepackage{ascmac}
\usepackage{systeme}
\usepackage{amsmath,cases}
\pagestyle{plain}

\newtheorem{theorem}{定理}[section]
\newtheorem{lemma}[theorem]{補題}
\newtheorem{proposition}[theorem]{命題}
\newtheorem{conjecture}[theorem]{予想}
\newtheorem{corollary}[theorem]{系}
\newtheorem{definition}[theorem]{定義}
\newtheorem{example}[theorem]{例}
\newtheorem{exercise}[theorem]{例題}
\newtheorem{problem}[theorem]{問}
\newtheorem{algorithm}[theorem]{アルゴリズム}
\newtheorem{remark}[theorem]{注意}

\def\qed{{\hfill$\square$}}
\def\proof{{\vspace{-0.3cm}f 証明: \,}}
\def\solution{{\vspace{-0.3cm}f 解: \,}}
\def\N{{\Bbb N}}
\def\Z{{\Bbb Z}}
\def\Q{{\Bbb Q}}
\def\R{{\Bbb R}}
\def\C{{\Bbb C}}
\def\F{{\Bbb F}}
\def\D{{\mathcal D}}
\def\mod{{\mathrm{mod\,\,}}}
\def\GL{{\mathrm{GL}}}
\def\GF{{\mathrm{GF}}}
\def\H{{\mathcal{H}}}

\setlength{\textwidth}{170mm}
\setlength{\textheight}{240mm}
\setlength{\oddsidemargin}{-5mm}
\setlength{\evensidemargin}{-5mm}
\setlength{\topmargin}{-10mm}
\setlength{\headheight}{0mm}
\setlength{\headsep}{10mm}

\title{項目反応理論}
\begin{document}
\maketitle
\setcounter{section}{2}
\section{尺度値の推定}
\setcounter{subsection}{3}
\subsection{ベイズ推定法}
\setcounter{subsubsection}{2}
\subsubsection{EAP推定法}
ベイズ推定法はMAP推定法ばかりではなくEAP法も利用される。これは、$\theta_i$の事後分布の$\theta_i$に関する期待値を推定値とする推定法である。式にすると、
\begin{align*}
  \label{00}
  \displaystyle \theta_i &= \int_{-\infty}^{+\infty} \theta_i \frac{g(\theta_i)L(\boldsymbol{u}_i|\theta)}{f(\boldsymbol{u}_i)}  d\theta_i  \\
  &= \frac{\displaystyle\int_{-\infty}^{+\infty} \theta_i g(\theta_i)L(\boldsymbol{u}_i|\theta) d\theta_i}{f(\boldsymbol{u}_i)} \\
  &= \frac{\displaystyle\int_{-\infty}^{+\infty} \theta_i g(\theta_i)L(\boldsymbol{u}_i|\theta) d\theta_i}{\displaystyle\int_{-\infty}^{+\infty}  g(\theta_i)L(\boldsymbol{u}_i|\theta) d\theta_i}\tag{3.35}
\end{align*}
となる。式中の積分は区分求積法によって求めることができる。

ある区間$[a,b]$を$N$等分した点$(幅は\Delta \theta)$を$\theta_n(n = 1,2,\cdots ,N,N+1)$とすると
\begin{align}
  \label{01}
  \displaystyle \int_{b}^{a} f(\theta) d\theta \simeq \sum_{n=1}^{N} f(\theta) \times \Delta \theta  \tag{3.36}
\end{align}
によって$N$個の長方形の和によって近似することができる。式(\ref{00})中の区間は$[-\infty,\infty]$であるが、$g(\theta)$は標準正規分布を仮定しているので、積分区間は$[-3,3]$や$[-4,4]$などで十分であると考えられる。式にすると、
\begin{align}
  \label{02}
  \displaystyle \hat{\theta}_i \simeq \frac{\displaystyle\sum_{n=1}^{N} \theta_{in} L(\boldsymbol{u}_i | \theta_{in})g(\theta_{in})\Delta \theta}{\displaystyle\sum_{n=1}^{N} L(\boldsymbol{u}_i | \theta_{in})g(\theta_{in})\Delta \theta}  \tag{3.37}
\end{align}
と近似できる。これまでの方法と違って、直接求めることができるので、反復計算の必要のないEAP推定法は計算時間を短縮することができる。
\section{項目母数の推定}
第$3$章では項目母数は既知であると仮定したうえで進めてきた。今回からは、反応パタン$\boldsymbol U$から項目母数を推定する方法について考える。
\subsection{同時最尤推定法}
本節では$\boldsymbol U$から項目母数と被験者母数を同時に推定する方法を紹介する。第$3$章で導入したように、被験者母数$\theta_i$が与えられたとき、$u_{ij}$の分布は、
\begin{align}
  \label{03}
  \displaystyle f(u_{ij}|\theta_i) = p_j(\theta_i)^{u_{ij}}q_j(\theta_i)^{1-u_{ij}} \tag{4.1}
\end{align}
であった。ただし、これまでと違って今回は項目母数も未知であるので、
\begin{align}
  \label{04}
  \displaystyle f(u_{ij}|\theta_i,a_j,b_j,c_j) = p_j(\theta_i)^{u_{ij}}q_j(\theta_i)^{1-u_{ij}} \tag{4.2}
\end{align}
と書き直す。ここでは$3$母数モデルを用いて説明していく。局所独立の仮定から、$n$個の項目に関する被験者$i$の反応パタンベクトル$\boldsymbol u_i$が観察される確率は、
\begin{align}
  \label{05}
  \displaystyle f(\boldsymbol u_i|\theta_i,\boldsymbol a,\boldsymbol  b,\boldsymbol  c) = \prod_{j = 1}^{n} f(u_{ij}|\theta_i,a_j,b_j,c_j) \tag{4.3}
\end{align}
と表現される。ここで、
\begin{align}
  \label{06}
  \displaystyle \boldsymbol{a}&=(a_1,a_2,\cdots,a_n)^{\prime} \tag{4.4}\\
  \boldsymbol{b}&= (b_1,b_2,\cdots,b_n)^{\prime}\tag{4.5}\\
  \boldsymbol{c}&= (c_1,c_2,\cdots,c_n)^{\prime}\tag{4.6}
\end{align}
である。
被験者の反応は互いに独立であるから、被験者母数と項目母数が与えられた条件の下で、$n$個の項目に対する$N$人の被験者の反応パタン行列$\boldsymbol{U}$が観察される確率は、
\begin{align}
  \label{07}
  \displaystyle \boldsymbol{f}(\boldsymbol{U}|)&=(a_1,a_2,\cdots,a_n)^{\prime} \tag{4.7}
\end{align}


\end{document}
