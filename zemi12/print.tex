\documentclass[10pt]{jarticle}
\usepackage{a4wide}
\usepackage{amsmath}%数学記号
\usepackage{amssymb}%数学記号
\usepackage{epsfig}%図
\usepackage{latexsym}
\usepackage{supertabular}
\usepackage{graphicx}
\usepackage{color}
\usepackage{ascmac}
\usepackage{multicol}
\usepackage{ascmac}
\usepackage{systeme}
\usepackage{amsmath,cases}
\usepackage{here}
\pagestyle{plain}

\newtheorem{theorem}{定理}[section]
\newtheorem{lemma}[theorem]{補題}
\newtheorem{proposition}[theorem]{命題}
\newtheorem{conjecture}[theorem]{予想}
\newtheorem{corollary}[theorem]{系}
\newtheorem{definition}[theorem]{定義}
\newtheorem{example}[theorem]{例}
\newtheorem{exercise}[theorem]{例題}
\newtheorem{problem}[theorem]{問}
\newtheorem{algorithm}[theorem]{アルゴリズム}
\newtheorem{remark}[theorem]{注意}

\def\qed{{\hfill$\square$}}
\def\proof{{\vspace{-0.3cm}f 証明: \,}}
\def\solution{{\vspace{-0.3cm}f 解: \,}}
\def\N{{\Bbb N}}
\def\Z{{\Bbb Z}}
\def\Q{{\Bbb Q}}
\def\R{{\Bbb R}}
\def\C{{\Bbb C}}
\def\F{{\Bbb F}}
\def\D{{\mathcal D}}
\def\mod{{\mathrm{mod\,\,}}}
\def\GL{{\mathrm{GL}}}
\def\GF{{\mathrm{GF}}}
\def\H{{\mathcal{H}}}

\setlength{\textwidth}{170mm}
\setlength{\textheight}{240mm}
\setlength{\oddsidemargin}{-5mm}
\setlength{\evensidemargin}{-5mm}
\setlength{\topmargin}{-10mm}
\setlength{\headheight}{0mm}
\setlength{\headsep}{10mm}

\title{項目反応実験テスト}
\begin{document}
\maketitle
\section{教育史}
次の文と関係の深い人物を記号で答えてください。
\begin{itemize}
  \item[(a)]アメリカの教育学者。プラグマティズムの立場から生産活動を基礎とする労作学校を主張・実施した。
\end{itemize}
\begin{center}
  $(1)$デューイ \ $(2)$フレーベル \ $(3)$シュプランガー \ $(4)$スペンサー
%スペンサー、フレーベル、デューイ、シュプランガー
\end{center}
\begin{itemize}
  \item[(b)]従来の「臨画帳」を写すだけの美術教育に反対し、自由画中心の教育を提唱した。
\end{itemize}
\begin{center}
  $(1)$木下竹次 \ $(2)$芦田恵之介 \ $(3)$鈴木三重吉 \ $(4)$山本鼎
%スペンサー、フレーベル、デューイ、シュプランガー
\end{center}
\section{教育原理}
次の文で間違っているものを答えてください。
\begin{itemize}
  \item[(a)]特別支援学校の設置義務は都道府県にある。
  \item[(b)]児童福祉司は、児童相談所長の命を受けて、児童の保護その他児童福祉に関する事項について相談に応じ、専門技術に基づき必要な指導を行う等児童福祉増進に努めるものとされている。
  \item[(c)]アメリカの教育学者で近代教授学の父であるとされるのがコメニウスであり、彼はその著書「大教授学」においてすべての民衆を対象にした一般教育を構想し、一斉授業も提唱した。
  \item[(d)]いじめが止んでいる状態というのは、その行為が止んでいる状態が少なくとも$3$カ月継続していることをいう。
  %コメニウスはチェコの人である。
\end{itemize}
\section{教育法規[1]}
次の選択肢の中から正しいものを答えてください。
\begin{itemize}
  \item[(a)]普通免許状は$18$歳未満のもの、高等学校を卒業しないものや禁錮以上の刑に処せられたものにも、条件次第では授与される。
  \item[(b)]教育職員検定は、受験者の人物、学力、実務および身体について、ランダムに選ばれた現職の教員が行う。
  \item[(c)]教員免許更新制は、教員として必要な資質を保持できるように、定期的に最新の知識技能を身につけることで社会の尊厳と信頼を得ることを目指すものである。
  \item[(d)]教員はすべて、免許取得$50$年後に免許更新講習を受けその課程を修了しなければ、免許の有効期間を更新できず免許状はその効力を失う。
  %答えは明らか。
\end{itemize}
\section{教育法規[2]}
次の空欄に当てはまる語句を選びなさい。
\begin{itemize}
  \item[(a)]すべて国民は、法律の定めるところにより、その保護する子女に(\ \ \ \ \ 1 \ \ \ \ )を受けさせる義務を負う。(日本国憲法)
\end{itemize}
\begin{center}
  $(1)$教育 \ \ \ \ \ $(2)$義務教育 \ \ \ \ \ $(3)$普通教育 \ \ \ \ \  $(4)$特別教育
%普通教育
\end{center}
\section{教育心理}
次の文は何について述べたものか答えてください。
\begin{itemize}
  \item[(a)]人間が健全な発達をするためには人間社会で多様な経験をすることが必要なことを示してくれた。ヴィクトールと名づけられたその子は、会話はほぼ不可能な状態であった。これは、神経言語学で言われるある期間を過ぎてしまったためだといわれている。この子を題材にフランスでは映画化されそのタイトルは当時、映画界を賑わせた。
  %問題文がわかりつらい。
\end{itemize}
\begin{center}
  $(1)$アヴェロンの野生児 \ \ \ \ \ $(2)$野生の少年 \ \ \ \ \ $(3)$言語獲得の臨界期 \ \ \ \ \  $(4)$フィリップピネル
\end{center}
\begin{itemize}
  \item[(b)]記憶のまとまりとしての単位
\end{itemize}
\begin{center}
  $(1)$メモリ \ \ \ \ \ $(2)$チャンク \ \ \ \ \ $(3)$ユニット \ \ \ \ \  $(4)$ミラー
\end{center}
\begin{itemize}
  \item[(c)]学習曲線を描いたときに、難易度によって途中で学習の成果が向上しなくなる現象
\end{itemize}
\begin{center}
  $(1)$スランプ \ \ \ \ \ $(2)$停滞減少 \ \ \ \ \ $(3)$負の強化 \ \ \ \ \  $(4)$プラトー
\end{center}
\begin{table}[H]
  \centering
    \begin{tabular}{||c||c||c||c||}
      \hline
      \hline
        \ \ \ \ \ \ \ \ \ \ \ \ \ \ \ \ \ \ \ \ \ \ \ \ \ &  \ \ \ \ \ \ \ \ \ \ \ \ \ \ \ \ \ \ \ \ \ \ \ \ \ & \ \ \ \ \ \ \ \ \ \ \ \ \ \ \ \ \ \ \ \ \ \ \ \ \ & \ \ \ \ \ \ \ \ \ \ \ \ \ \ \ \ \ \ \ \ \ \ \ \ \  \\
      \hline
      \hline
        \ \ \ \ \ \ \ \ \ \ \ \ \ \ \ \ \ \ \ \ \ \ \ \ \ &  \ \ \ \ \ \ \ \ \ \ \ \ \ \ \ \ \ \ \ \ \ \ \ \ \ & \ \ \ \ \ \ \ \ \ \ \ \ \ \ \ \ \ \ \ \ \ \ \ \ \ & \ \ \ \ \ \ \ \ \ \ \ \ \ \ \ \ \ \ \ \ \ \ \ \ \  \\
      \hline
      \hline
    \end{tabular}
  \end{table}

ご協力ありがとうございました。




\end{document}
