\documentclass[12pt]{jarticle}
\usepackage{a4wide}
\usepackage{amsmath}%数学記号
\usepackage{amssymb}%数学記号
\usepackage{epsfig}%図
\usepackage{latexsym}
\usepackage{supertabular}
\usepackage{graphicx}
\usepackage{color}
\usepackage{ascmac}
\usepackage{multicol}
\usepackage{ascmac}
\usepackage{systeme}
\usepackage{amsmath,cases}
\pagestyle{plain}

\newtheorem{theorem}{定理}[section]
\newtheorem{lemma}[theorem]{補題}
\newtheorem{proposition}[theorem]{命題}
\newtheorem{conjecture}[theorem]{予想}
\newtheorem{corollary}[theorem]{系}
\newtheorem{definition}[theorem]{定義}
\newtheorem{example}[theorem]{例}
\newtheorem{exercise}[theorem]{例題}
\newtheorem{problem}[theorem]{問}
\newtheorem{algorithm}[theorem]{アルゴリズム}
\newtheorem{remark}[theorem]{注意}

\def\qed{{\hfill$\square$}}
\def\proof{{\vspace{-0.3cm}f 証明: \,}}
\def\solution{{\vspace{-0.3cm}f 解: \,}}
\def\N{{\Bbb N}}
\def\Z{{\Bbb Z}}
\def\Q{{\Bbb Q}}
\def\R{{\Bbb R}}
\def\C{{\Bbb C}}
\def\F{{\Bbb F}}
\def\D{{\mathcal D}}
\def\mod{{\mathrm{mod\,\,}}}
\def\GL{{\mathrm{GL}}}
\def\GF{{\mathrm{GF}}}
\def\H{{\mathcal{H}}}

\setlength{\textwidth}{170mm}
\setlength{\textheight}{240mm}
\setlength{\oddsidemargin}{-5mm}
\setlength{\evensidemargin}{-5mm}
\setlength{\topmargin}{-10mm}
\setlength{\headheight}{0mm}
\setlength{\headsep}{10mm}

\title{項目反応理論}
\begin{document}
\maketitle
\section{項目特性曲線とは}
項目特性曲線について復習しておく。項目特性曲線とは、(流儀は多々あるが今回は)横軸に能力特性で分けられた群番号、縦軸には郡ごとの正答率を付したグラフである。グラフの形によっていくつかの型に分けられた。ここで、能力特性で群に分けるとは合計得点をもとに高いものから$5$群、低い者たちを$1$群とし振り分けることである。型については以下の通りである。%ゼミプリントから持ってくる
\section{実験}
\subsection{概要}
教職教養に関する問題を$8$問出題し、その回答を項目特性図によって、誤答分析した後に、各選択肢をみて正答分析を行ってみる。
\subsection{条件設定}
教職教養に関する問題ということで、なるべくこれに関して勉強を行っているほうがよいと考え、教育学部に在籍しているもしくは在籍していたもの$30$人を対象に行った。人数に関しては実際の学校現場における$1$教室の人数となるべく近いようにするために今回は$30$人とした。出題する問題は、教職教養の勉強に広く用いられる時事通信社の「教職教養の演習問題」から$8$題出題する。
\subsection{問題に関して}
第$1$問、第$2$問は教育史から出題した。出題された文に関する人物を選ぶ問題である。第$1$問は第$2$問に比べて比較的基礎的な問題と予想されるので、項目特性曲線は、型になると予想される。一方で、第$2$問は九州の中の県では、あまり出題されない日本教育史であるため、正答率は群に関係なく低くなると考えられるため、型になると予想される。

第$3$問は教育原理から出題した。難易度は高く設定された問題である。理由としては、一見したら全部正しいように見えるが、実は間違っているのはコメニウスはアメリカの学者ではなくチェコの学者であるという点である。これは実際に出された問題であり、ほとんど消去法で国の違いにたどり着くか、運で正答するかという問題である。このことを考えると型になると予想できる。

第$4$問、第$5$問は教育法規から出題した。第$4$問は、正解の選択肢以外はほとんど間違いようのない選択肢ばかりである。つまりどの群においても正答率は高くなると考えられるので、型になると予想される。第$5$問は難易度は比較的簡単に設定されている。しかし、$2$択までは簡単に絞れるがそこからしっかりと正答を導けるのかが大事になってくる問題である。このことから、高位群では安定の正答率が出て、下位群になればなるほど不安定になると考えられる。よって。型になると予想される。

第$6$問、第$7$問、第$8$問は、教育心理から出題した。第$6$問は問題文が理解しづらい問題である。選択がバラバラになると考えられるため、型になると予想される。第$7$問、第$8$問は、記憶していればしっかりと答えられる問題である。

\section{正答分析}
\section{誤答分析}










\end{document}
