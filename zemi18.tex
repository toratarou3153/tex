\documentclass[20pt]{jarticle}
\usepackage{a4wide}
\usepackage{amsmath}%数学記号
\usepackage{amssymb}%数学記号
\usepackage{epsfig}%図
\usepackage{latexsym}
\usepackage{supertabular}
\usepackage{graphicx}
\usepackage{color}
\usepackage{ascmac}
\usepackage{multicol}
\usepackage{ascmac}
\usepackage{systeme}
\usepackage{amsmath,cases}
\usepackage{float}
\pagestyle{plain}

\newtheorem{theorem}{定理}[section]
\newtheorem{lemma}[theorem]{補題}
\newtheorem{proposition}[theorem]{命題}
\newtheorem{conjecture}[theorem]{予想}
\newtheorem{corollary}[theorem]{系}
\newtheorem{definition}[theorem]{定義}
\newtheorem{example}[theorem]{例}
\newtheorem{exercise}[theorem]{例題}
\newtheorem{problem}[theorem]{問}
\newtheorem{algorithm}[theorem]{アルゴリズム}
\newtheorem{remark}[theorem]{注意}

\def\qed{{\hfill$\square$}}
\def\proof{{\vspace{-0.3cm}f 証明: \,}}
\def\solution{{\vspace{-0.3cm}f 解: \,}}
\def\N{{\Bbb N}}
\def\Z{{\Bbb Z}}
\def\Q{{\Bbb Q}}
\def\R{{\Bbb R}}
\def\C{{\Bbb C}}
\def\F{{\Bbb F}}
\def\D{{\mathcal D}}
\def\mod{{\mathrm{mod\,\,}}}
\def\GL{{\mathrm{GL}}}
\def\GF{{\mathrm{GF}}}
\def\H{{\mathcal{H}}}

\setlength{\textwidth}{170mm}
\setlength{\textheight}{240mm}
\setlength{\oddsidemargin}{-5mm}
\setlength{\evensidemargin}{-5mm}
\setlength{\topmargin}{-10mm}
\setlength{\headheight}{0mm}
\setlength{\headsep}{10mm}

\title{項目反応理論}
\begin{document}
\maketitle
\section{多次元項目反応理論}
\subsection{被験者の特性を線形結合で表現}
$2$パラメータで考える。

$k$次元の特性を測定する多次元項目反応モデルにおいて、$j$個の項目に対する反応が特性値を所与とすると互いに独立のとき、特性値を所与とした尤度は
\begin{align}
  \label{01}
  \displaystyle
    L(u_1,u_2,\cdots,u_j|\theta = (\theta_1,\cdots,\theta_K)^{t})=\prod_j^J P_j(\theta)^{u_j} Q_j(\theta)^{(1 - u_j)}
  \tag{1}
\end{align}
ただし、$u_j$は項目$j$に対して正答するか誤答するかを表す確率変数であり、$\theta_k$は特性$k$についての特性値である。

ここで、特性値$\theta$の分布を$p(\theta)$とおくと、周辺尤度$L(u_1,u_2,\cdots,u_j)$は
\begin{align}
  \label{02}
  \displaystyle
    L(u_1,u_2,\cdots,u_j) = \int L(u_1,u_2,\cdots,u_j|\theta)p(\theta) d \theta
  \tag{2}
\end{align}
またここで、$P_j(\theta)$は$\theta$を所与とした場合の正答確率である。これを$2$パラメータのロジスティックモデルで表現すると
\begin{align}
  \label{03}
  \displaystyle
    P_j(\theta) = \frac{\exp D (\sum\limits_{k = 1}^{K} w_{jk} \theta_k + b_j)}{1 + \exp D (\sum\limits_{k = 1}^{K} w_{jk} \theta_k + b_j)}
  \tag{3}
\end{align}である。ここで、$w_{jk}$は第$j$項目の第$k$特性の識別力パラメータであり、$b_j$は第$j$項目の困難度パラメータ、$D=1.8$である。

\subsection{アルゴリズム}
被験者:$i = 1,2,3,4$

項目:$j = 1,2,3,4,5$

能力パラメータ:$\theta_{jk}=(\theta_{j1},\theta_{j2})$
\begin{figure}[H]
  \centering
  \begin{tabular}{|c|c|c|c|c|} \hline
     & A & B & C & D \\ \hline
    1 & 1 & 0 & 0 & 0 \\ \hline
    2 & 0 & 1 & 0 & 0 \\ \hline
    3 & 0 & 0 & 1 & 0 \\ \hline
    4 & 0 & 0 & 0 & 1 \\ \hline
    5 & 1 & 1 & 1 & 1 \\ \hline
  \end{tabular}
  \caption{反応データ}
\end{figure}

困難度:$b_j = (-2, -1, 0, 1, 2)$

項目ごとの能力パラメータの識別力:
\begin{align}
  \label{04}
  \displaystyle
    a_{jk} =
    \begin{pmatrix}
      a_{11} & a_{12} \\
      a_{21} & a_{22} \\
      a_{31} & a_{32} \\
      a_{41} & a_{42} \\
      a_{51} & a_{52} \\
    \end{pmatrix}
  \tag{4}
\end{align}

$2$パラメータのロジスティックモデルを次の式に仮定する。
\begin{align}
  \label{05}
  \displaystyle
    P_j (\theta) = \frac{\exp\{D(\sum\limits_{k = 1}^{K} a_{jk}\theta_jk + b_j) \}}{1 + \exp\{D(\sum\limits_{k = 1}^{K} a_{jk}\theta_jk + b_j)\}}
  \tag{5}
\end{align}

例えば、項目$j=1$に被験者$i=1$が正解する確率は,
\begin{align*}
  \label{06}
  \displaystyle
  P_j (\theta) &= \frac{\exp\{(-1.7)(\sum\limits_{k = 1}^{2} a_{jk}\theta_jk + b_1) \}}{1 + \exp\{(-1.7)(\sum\limits_{k = 1}^{2} a_{jk}\theta_jk + b_1)\}}\\
  &= \frac{\exp\{(-1.7)(a_{11} \theta_{11} + a_{12}\theta_{12} + b_1) \}}{1 + \exp\{(-1.7)(a_{11} \theta_{11} + a_{12}\theta_{12} + b_1)\}}
  \tag{6}
\end{align*}
となる。これを用いて反応行列から正答のときは$P_j(\theta)$,誤答のときは$1 - P_j(\theta)$として、それぞれの尤度関数を求めていく。例えば能力値を推定する場合は$\theta$が変数となる。$\theta$は今回$2$種類を考えているので尤度を$\theta_jk$で偏微分してから最尤法で最大となる$\theta$の組を求めていく。
\end{document}
