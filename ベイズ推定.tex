\documentclass[12pt]{jarticle}
\usepackage{a4wide}
\usepackage{amsmath}%数学記号
\usepackage{amssymb}%数学記号
\usepackage{epsfig}%図
\usepackage{latexsym}
\usepackage{supertabular}
\usepackage{graphicx}
\usepackage{color}
\usepackage{ascmac}
\usepackage{multicol}
\usepackage{ascmac}
\usepackage{systeme}
\usepackage{amsmath,cases}
\pagestyle{plain}

\newtheorem{theorem}{定理}[section]
\newtheorem{lemma}[theorem]{補題}
\newtheorem{proposition}[theorem]{命題}
\newtheorem{conjecture}[theorem]{予想}
\newtheorem{corollary}[theorem]{系}
\newtheorem{definition}[theorem]{定義}
\newtheorem{example}[theorem]{例}
\newtheorem{exercise}[theorem]{例題}
\newtheorem{problem}[theorem]{問}
\newtheorem{algorithm}[theorem]{アルゴリズム}
\newtheorem{remark}[theorem]{注意}

\def\qed{{\hfill$\square$}}
\def\proof{{\vspace{-0.3cm}f 証明: \,}}
\def\solution{{\vspace{-0.3cm}f 解: \,}}
\def\N{{\Bbb N}}
\def\Z{{\Bbb Z}}
\def\Q{{\Bbb Q}}
\def\R{{\Bbb R}}
\def\C{{\Bbb C}}
\def\F{{\Bbb F}}
\def\D{{\mathcal D}}
\def\mod{{\mathrm{mod\,\,}}}
\def\GL{{\mathrm{GL}}}
\def\GF{{\mathrm{GF}}}
\def\H{{\mathcal{H}}}

\setlength{\textwidth}{170mm}
\setlength{\textheight}{240mm}
\setlength{\oddsidemargin}{-5mm}
\setlength{\evensidemargin}{-5mm}
\setlength{\topmargin}{-10mm}
\setlength{\headheight}{0mm}
\setlength{\headsep}{10mm}

\title{項目反応理論}
\begin{document}
\maketitle
\section{ベイズ推定}
\subsection{MAP推定法}
$\theta_1$(被験者$1$の能力値)の値が$a$以上$b$以下である確率を
\begin{align*}
  \label{00}
  \displaystyle P(a \leq \theta_1 \leq b) &= \int_{a}^{b} \pi(x) dx \\ &= P(a \leq X \leq b)
  \tag{1}
\end{align*}
とする。ここでは、
\begin{align*}
  \label{01}
  \displaystyle \pi(x) = \frac{1}{\sqrt{2\pi}}\exp\left(- \frac{x^2}{2}\right)
  \tag{2}
\end{align*}と仮定する。
式$(\ref{01})$を対数に変換すると、
\begin{align}
  \label{02}
  \displaystyle \log \pi(x) = \log\frac{1}{\sqrt{2\pi}} - \frac{x^2}{2} \tag{3}
\end{align}
ここで確率変数が連続型である時にベイズの定理を適応すると、
\begin{align}
  \label{03}
  \displaystyle \pi(x|\boldsymbol{u}_1) = \frac{L(\boldsymbol{u}_1)\pi(x)}{\int_{-\infty}^{\infty}L(\boldsymbol{u}_1)\pi(x)dx}
  \tag{4}
\end{align}となり、これを対数変換したものを最大にする$x$の値をMAP推定値という。
\begin{align}
  \label{04}
  \displaystyle  \log \pi(x|\boldsymbol{u}_1) &= \log L(\boldsymbol{u}_1) + \log \pi(x) - \log\int_{-\infty}^{\infty}L(\boldsymbol{u}_1)\pi(x)dx \notag\\
  &= \log L(\boldsymbol{u}_1) -\frac{x^2}{2} + C
  \tag{5}
\end{align}
\subsection{アルゴリズム}
\begin{itemize}
  \item[\textbf{step1}] それぞれの尤度関数を求める。
  \item[\textbf{step2}] 事前分布$\pi(x)$を定める。
  \item[\textbf{step3}] $\pi(x|\boldsymbol{u}_1)$を求める。
  \item[\textbf{step4}] $\log\pi(x|\boldsymbol{u}_1)$を最大にする$x$を求める。
\end{itemize}





\end{document}
