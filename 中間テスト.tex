\documentclass[twocolumn, 20pt]{jarticle}
\usepackage{a4wide}
\usepackage{amsmath}%数学記号
\usepackage{amssymb}%数学記号
\usepackage{epsfig}%図
\usepackage{latexsym}
\usepackage{supertabular}
\usepackage{graphicx}
\usepackage{color}
\usepackage{ascmac}
\usepackage{ascmac}
\usepackage{systeme}
\usepackage{amsmath,cases}
\pagestyle{plain}

\newtheorem{theorem}{定理}[section]
\newtheorem{lemma}[theorem]{補題}
\newtheorem{proposition}[theorem]{命題}
\newtheorem{conjecture}[theorem]{予想}
\newtheorem{corollary}[theorem]{系}
\newtheorem{definition}[theorem]{定義}
\newtheorem{example}[theorem]{例}
\newtheorem{exercise}[theorem]{例題}
\newtheorem{problem}[theorem]{問}
\newtheorem{algorithm}[theorem]{アルゴリズム}
\newtheorem{remark}[theorem]{注意}

\def\qed{{\hfill$\square$}}
\def\proof{{\vspace{-0.3cm}f 証明: \,}}
\def\solution{{\vspace{-0.3cm}f 解: \,}}
\def\N{{\Bbb N}}
\def\Z{{\Bbb Z}}
\def\Q{{\Bbb Q}}
\def\R{{\Bbb R}}
\def\C{{\Bbb C}}
\def\F{{\Bbb F}}
\def\D{{\mathcal D}}
\def\mod{{\mathrm{mod\,\,}}}
\def\GL{{\mathrm{GL}}}
\def\GF{{\mathrm{GF}}}
\def\H{{\mathcal{H}}}

\setlength{\textwidth}{170mm}
\setlength{\textheight}{265mm}
\setlength{\oddsidemargin}{-5mm}
\setlength{\evensidemargin}{-5mm}
\setlength{\topmargin}{-10mm}
\setlength{\headheight}{0mm}
\setlength{\headsep}{1mm}
\setlength{\columnsep}{5mm}

\renewcommand{\baselinestretch}{0.01}

\title{\large{$2022$(R4年度) S2 プログレスクラス 数学$\textcircled{1}$ 1学期中間考査} 問題用紙\\ $\leftline{\Large{クラス(\ \ \ )\ \ 番号(\ \ \ )\ \ 名前(\ \ \ \ \ \ \ \ \ \ \ \ \ \ \ \ \ \ \ )}}$}
\begin{document}
\columnseprule=0.2mm
\twocolumn[\date{}
\maketitle
\vspace{-0.7cm}
\rule{\textwidth}{0.3mm}
\vspace{0.2cm}]
\section{}
次の複素数の実部と虚部を言いなさい。

$(1)-4+2i$

$(2)\displaystyle \frac{3-\sqrt{5}i}{2}$

$(3)3i$
\vspace{-0.5cm}
\section{}
次の計算をせよ。

$(1)\displaystyle \frac{1+7i}{1+2i}$

$(2)\displaystyle \sqrt{-3}\sqrt{-6}$

$(3)\displaystyle \frac{\sqrt{-12}}{\sqrt{-3}}$
\vspace{-0.5cm}
\section{}
次の$2$点間の距離を求めなさい。

$(1)(-2, 5),(4,7)$

$(2) (-3, -3), (0,0)$
\vspace{-0.5cm}
\section{}
点$A(2, 3)$に関して点$P(-1,4)$と対称な点$Q$の座標を求めよ。
\vspace{-0.5cm}
\section{}
次のような直線の方程式を求めよ。

$(1)$点$(3,4)$を通り、傾きが$2$の直線

$(2)$$2$点$(2, 3), (4, 5)$を通る直線
\vspace{-0.5cm}
\section{}
次の等式を満たす実数$\displaystyle x,y$を求めなさい。

$ x(1+2i)^2 + y(1-i)^2 + 6 = 0$
\vspace{-0.5cm}
\section{}
$x$は実数とする。$\displaystyle \alpha = \frac{x-i}{2+i}$について$\displaystyle \alpha$が実数となるとき$x$の値を求めよ。

\vspace{-0.5cm}
\section{}

次の$2$次方程式を解きなさい。

$\displaystyle (1)x^2 + 2x + 5=0$

$\displaystyle(2) 3x^2 - 4x -1 =0$
\vspace{-0.5cm}
\section{}

$2$つの$2$次方程式$x^2 + ax +a +3 =0$,\ \ $x^2 -ax + 4 = 0$がともに虚数解をもつとき、定数$a$の値の範囲を求めよ。
\vspace{-0.5cm}
\section{}

$2$次方程式$x^2 + 2ax + a +2 = 0,\ \ x^2 -4x + a + 3=0$の少なくとも一方が実数解をもつような、定数$a$の値の範囲を求めよ。
\vspace{-0.5cm}
\section{}

$2$次方程式$x^2 + 6x -3 = 0$の$2$つの解を$\alpha,\beta$とするとき、次の式の値を求めなさい。

$\displaystyle (1) \alpha^2 + \beta^2$

$\displaystyle (2) \alpha^2\beta + \alpha\beta^2$

$\displaystyle (3) \alpha^3 + \beta^3$

\vspace{-0.5cm}
\section{}
$2$次方程式$x^2 + ax + b=0$の$2$つの解を$\alpha,\beta$とする。$\alpha+\beta,\alpha\beta$を解とする$2$次方程式が$x^2 + 2ax + b + 2=0$のとき、定数$a,b$の値を求めなさい。


\vspace{-0.5cm}
\section{}
次の式を因数分解しなさい。

$(1)x^3 - 9x^2 + 27x - 27$

$(2)3x^3-8x^2 -15x -4$
\vspace{-0.5cm}
\section{}
整式$P(x)$を$x+5$で割った余りが$8$で、$x-2$で割った余りが$4$であるとき、$P(x)$を$(x+5)(x-2)$で割った余りを求めよ。
\vspace{-0.5cm}
\section{}

立方体の縦を$2cm$、横を$3cm$それぞれ伸ばして、高さを$1cm$縮めて直方体を作ったら、体積が$60cm^3$となった、このとき、元の立方体の一片の長さを求めよ。
\vspace{-0.5cm}
\section{}

$3$次方程式$x^3 - (a + 3)x^2+9a=0$について、この方程式が$3$を$2$重解としてもつように、定数$a$の値を求めなさい。

\vspace{-0.5cm}
\section{}
$2$次方程式$x^2 + 2(3m - 1)x + 9m^2 -4 =0 $が、異なる$2$つの負の解をもつとき、定数$m$の値の範囲を求めなさい。


\vspace{-0.5cm}
\section{}
$2$点$(-4, -5),(8, 1)$を通る直線を$\displaystyle l$とする。

$(1)$直線$\displaystyle l$の方程式を求めなさい。

$(2)$点$(2, -3)$を通り、$\displaystyle l$に垂直な直線の方程式を求めなさい。
\vspace{-0.5cm}
\section{}
$2$点$A(1, 3),B(5, 1)$を結ぶ線分の垂直$2$等分線の方程式を求めよ。

\end{document}
