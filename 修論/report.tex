\documentclass[12pt]{jarticle}
\usepackage{a4wide}
\usepackage{amsmath}%数学記号
\usepackage{amssymb}%数学記号
\usepackage{epsfig}%図
\usepackage{latexsym}
\usepackage{supertabular}
\usepackage{graphicx}
\usepackage{color}
\usepackage{ascmac}
\usepackage{multicol}
\usepackage{ascmac}
\usepackage{systeme}
\usepackage{amsmath,cases}
\usepackage{float}
\usepackage{here}
\pagestyle{plain}


\newtheorem{theorem}{定理}[section]
\newtheorem{lemma}[theorem]{補題}
\newtheorem{proposition}[theorem]{命題}
\newtheorem{conjecture}[theorem]{予想}
\newtheorem{corollary}[theorem]{系}
\newtheorem{definition}[theorem]{定義}
\newtheorem{example}[theorem]{例}
\newtheorem{exercise}[theorem]{例題}
\newtheorem{problem}[theorem]{問}
\newtheorem{algorithm}[theorem]{アルゴリズム}
\newtheorem{remark}[theorem]{注意}

\def\qed{{\hfill$\square$}}
\def\proof{{\vspace{-0.3cm}f 証明: \,}}
\def\solution{{\vspace{-0.3cm}f 解: \,}}
\def\N{{\Bbb N}}
\def\Z{{\Bbb Z}}
\def\Q{{\Bbb Q}}
\def\R{{\Bbb R}}
\def\C{{\Bbb C}}
\def\F{{\Bbb F}}
\def\D{{\mathcal D}}
\def\mod{{\mathrm{mod\,\,}}}
\def\GL{{\mathrm{GL}}}
\def\GF{{\mathrm{GF}}}
\def\H{{\mathcal{H}}}

\setlength{\textwidth}{170mm}
\setlength{\textheight}{240mm}
\setlength{\oddsidemargin}{-5mm}
\setlength{\evensidemargin}{-5mm}
\setlength{\topmargin}{-10mm}
\setlength{\headheight}{0mm}
\setlength{\headsep}{10mm}

\title{「歴史に学ぶ数学」レポート}
\begin{document}
\date{}
\author{210-d8551 西郷虎太郎}
\maketitle
\section{レポート問題$3$}
$\mathbb{H}$上の変換$\displaystyle z \rightarrow \frac{ax+b}{cx+d}$$s.t. a, b, c, d\in \mathbb{R}, ad-bc=1$は$P$計量に関する等長変換であることを示せ。\\
\\
\\
$\displaystyle w=\frac{az+b}{cz+d}$として$\displaystyle \frac{dwd\bar{w}}{w - \bar{w}} = \frac{dzd\bar{z}}{(z - \bar{z})^2}$になることを示す。

\begin{align}
  \label{}
  \displaystyle
  \frac{dw}{dz} = \frac{(az+b)^{\prime}(cz+b)-(az+b)(cz+d)^{\prime}}{(cz+d)^2} = \frac{dz}{(cz+d)^2}
\end{align}
また、
\begin{align}
  \label{}
  \displaystyle
  \frac{d\bar{w}}{dz} = \frac{d\bar{z}}{(c\bar{z} + d)^{\prime}}
\end{align}

ここで、$w - \bar{w}$を考えると,
\begin{align}
  \label{}
  \displaystyle
  w - \bar{w} = \frac{(az+b)(c\bar{z}+b)-(a\bar{z}+b)(cz+d)}{(cz+d)(c\bar{z}+d)} = \frac{z - \bar{z}}{(cz+d)(c\bar{z}+d)}
\end{align}
よって
\begin{align}
  \label{}
  \displaystyle
  (左辺) = \frac{dwd\bar{w}}{w - \bar{w}}= \frac{dzd\bar{z}}{(z - \bar{z})^2} = (右辺)
\end{align}
\newpage
\section{レポート問題$4$}
斜線で示された部分の面積を求めなさい。
\\
\\
求める部分の面積を$\boldsymbol{S}$とすると
\begin{align*}
  \label{}
  \displaystyle
  \boldsymbol{S} = \int_D \frac{1}{y^2}dxdy &= \int_{-1}^1 dx \int_{\sqrt{1 - x^2}}^\infty \frac{1}{y^2} dy \\ &= \int_{-1}^1 \lim_{a \to \infty} \int_{\sqrt{1 - x^2}}^a \frac{1}{y^2} dy \\ &= \int_{-1}^1 \lim_{a \to \infty} \{-\frac{1}{a} - (-\frac{1}{\sqrt{1 - x^2}})\}dx  \\ &= \int_{-1}^1 \frac{1}{\sqrt{1 - x^2}}dx \\ ここで x = \sin{\theta} と置くと \frac{dx}{d\theta} = \cos{\theta}より\\
  &= \int_{-\frac{\pi}{2}}^{\frac{\pi}{2}} \frac{1}{\cos{\theta}}\cos{\theta} d\theta \\ &= \int_{-\frac{\pi}{2}}^{\frac{\pi}{2}} d\theta = \frac{\pi}{2} + \frac{\pi}{2} = \pi
\end{align*}
よって$\boldsymbol{S} = \pi$
\end{document}
