\documentclass[twocolumn, 20pt]{jarticle}
\usepackage{a4wide}
\usepackage{amsmath}%数学記号
\usepackage{amssymb}%数学記号
\usepackage{epsfig}%図
\usepackage{latexsym}
\usepackage{supertabular}
\usepackage{graphicx}
\usepackage{color}
\usepackage{ascmac}
\usepackage{ascmac}
\usepackage{systeme}
\usepackage{amsmath,cases}
\pagestyle{plain}

\newtheorem{theorem}{定理}[section]
\newtheorem{lemma}[theorem]{補題}
\newtheorem{proposition}[theorem]{命題}
\newtheorem{conjecture}[theorem]{予想}
\newtheorem{corollary}[theorem]{系}
\newtheorem{definition}[theorem]{定義}
\newtheorem{example}[theorem]{例}
\newtheorem{exercise}[theorem]{例題}
\newtheorem{problem}[theorem]{問}
\newtheorem{algorithm}[theorem]{アルゴリズム}
\newtheorem{remark}[theorem]{注意}

\def\qed{{\hfill$\square$}}
\def\proof{{\vspace{-0.3cm}f 証明: \,}}
\def\solution{{\vspace{-0.3cm}f 解: \,}}
\def\N{{\Bbb N}}
\def\Z{{\Bbb Z}}
\def\Q{{\Bbb Q}}
\def\R{{\Bbb R}}
\def\C{{\Bbb C}}
\def\F{{\Bbb F}}
\def\D{{\mathcal D}}
\def\mod{{\mathrm{mod\,\,}}}
\def\GL{{\mathrm{GL}}}
\def\GF{{\mathrm{GF}}}
\def\H{{\mathcal{H}}}

\setlength{\textwidth}{170mm}
\setlength{\textheight}{240mm}
\setlength{\oddsidemargin}{-5mm}
\setlength{\evensidemargin}{-5mm}
\setlength{\topmargin}{-10mm}
\setlength{\headheight}{0mm}
\setlength{\headsep}{10mm}
\setlength{\columnsep}{5mm}

\title{模擬中間テスト}
\begin{document}
\columnseprule=0.2mm
\date{}
\maketitle
\vspace{-2cm}
\subsection{次の値を求めなさい}
\begin{itemize}
  \item [(1)] $ \displaystyle 10^0$
  \item [(2)] $\displaystyle (-5)^{-3}$
  \item [(3)] $\displaystyle {0.5}^{-4}$
  \item [(4)] $\displaystyle a^{-3}a^{7}$
  \item [(5)] $\displaystyle (a^2)^{-3} \div (a^{-1})^5$
  \item [(6)] $\displaystyle \sqrt[4]{16}$
  \item [(7)] $\displaystyle \sqrt[3]{\frac{1}{8}}$
  \item [(8)] $\displaystyle \sqrt[4]{3}\sqrt[4]{27}$
\end{itemize}

\subsection{$2^x - 2^{-x} = 1$のとき$4^x + 4^{-x}$と$8^x - 8^{-x}$の値を求めなさい}


\subsection{次の方程式、不等式を解きなさい}
$\displaystyle (1) 4^x = 64$
\\
\\
$\displaystyle (2) 8^{2x+3} = 2^{3x + 5}$
\\
\\
$\displaystyle (3) (\frac{1}{9})^{3x-1} = (\frac{1}{27})^{x}$
\\
\\
$\displaystyle (4) 2^x < 32$
\\
\\
$\displaystyle (5) (\frac{1}{4})^x \geqq \frac{1}{8}$
\\
\\
$\displaystyle (6) (\frac{1}{36})^{2x-1} \geqq 6^{x+3}$

\subsection{次の値を求めなさい}
\begin{itemize}
  \item [(1)] $ \displaystyle log_3 9$
  \item [(2)] $\displaystyle log_5 5$
  \item [(3)] $\displaystyle log_7 1$
  \item [(4)] $\displaystyle log_6 3 + log_6 12$
  \item [(5)] $\displaystyle 2log_3 \sqrt{3} - \frac{1}{2}log_3 6 + log_3 \frac{\sqrt{6}}{3}$
  \item [(6)] $\displaystyle log_2 \frac{2}{3} + log_2 24$
  \item [(7)] $\displaystyle log_5 10 - log_5 2\sqrt{5}$
\end{itemize}

\subsection{$log_2 3 =a, log_2 5 = b$のとき、$\displaystyle log_2 75$をa,bを用いて表しなさい}
\subsection{次の方程式、不等式を解きなさい}
\begin{itemize}
  \item [(1)] $ \displaystyle log_{0.2}x=-2$
  \item [(2)] $\displaystyle log_2 (x+1)=3$
  \item [(3)] $\displaystyle log_{27}x>\frac{1}{3}$
  \item [(4)] $\displaystyle log_3(x-3)+log_3(2x+1)=2$
  \item [(5)] $\displaystyle log_3(x-4)+log_3 (x-2)<1$
\end{itemize}

\subsection{$3^{40}$は何桁の数か求めなさい。ただし、$log_10 2 =0.3010,log_10 3 = 0.4771$とする。}
\end{document}
