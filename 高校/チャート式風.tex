\documentclass[12pt]{jarticle}
\usepackage{fancyhdr}
\usepackage{amsmath}
\usepackage{multicol}
\usepackage{tcolorbox}
\usepackage{ascmac}
\usepackage{comment}
\usepackage{enumerate}
\usepackage{fancybox}
\usepackage{framed}
\usepackage{setspace}
\usepackage{tikz}
\usepackage{enumitem}
\usepackage{tasks}
\usepackage{ifthen}
\usepackage[top=28mm, bottom=28mm, left=17mm,right=17mm]{geometry} %%%%%%%%%%%%%%設定色々%%%%%%%%%%%%%%%
\pagestyle{fancy}
\cfoot{}
\renewcommand{\headrulewidth}{0pt}
\setlength{\columnseprule}{0.5pt}
\setlength{\columnwidth}{5mm}
\newcommand{\tanngenn}[1]{\lhead{\framebox[176mm][l]{#1}}}
\newenvironment{reidai}[1]{\noindent{\Large\colorbox[gray]{0.8}{基}\fcolorbox{white}{gray}{\textcolor{white}{本}}\colorbox[gray]{0.8}{例}\fcolorbox{white}{gray}{\textcolor{white}{題}}} {\large \hspace{2mm} #1 \hspace{2mm}}\hrulefill\\ \par }{\par \noindent\hrulefill\\}
\newcommand{\sisinn}[1]{\noindent{\bf 指針} #1}
\newcommand{\rennsyuu}{\noindent\fcolorbox{white}{gray}{\textcolor{white}{練}}\colorbox[gray]{0.8}{習}\fcolorbox{white}{gray}{\textcolor{white}{問}}\colorbox[gray]{0.8}{題}}
\newenvironment{kaitou}{\noindent{\bf 解答} \dotfill}{\noindent\dotfill\\}
%%%%%%%%%%%%%%ここから本文開始%%%%%%%%%%%%%%%
\begin{document}
\tanngenn{ここに単元名を記入}
\begin{reidai}{ここにタイトル} %タイトルがない場合空欄にしておく
\noindent ここに基本例題を記入。ここに基本例題を記入。ここに基本例題を記入。ここに基本例題を記入。ここに基本例題を記入。ここに基本例題を記入。ここに基本例題を記入。ここに基本例題を記入。ここに基本例題を記入。ここに基本例題を記入。ここに基本例題を記入。ここに基本例題を記入。ここに基本例題を記入。ここに基本例題を記入。ここに基本例題を記入。ここに基本例題を記入。ここに基本例題を記入。ここに基本例題を記入。
\end{reidai}

\sisinn{ここに指針タイトルを記入}
\begin{itemize}
\item 指針を記入
\item 指針を記入\\
\end{itemize}

\begin{kaitou}
\begin{multicols}{2}
\noindent ここに解答を記入。ここに解答を記入。ここに解答を記入。ここに解答を記入。ここに解答を記入。ここに解答を記入。ここに解答を記入。ここに解答を記入。ここに解答を記入。ここに解答を記入。ここに解答を記入。ここに解答を記入。ここに解答を記入。ここに解答を記入。ここに解答を記入。ここに解答を記入。ここに解答を記入。ここに解答を記入。ここに解答を記入。ここに解答を記入。ここに解答を記入。ここに解答を記入。ここに解答を記入。ここに解答を記入。ここに解答を記入。ここに解答を記入。ここに解答を記入。ここに解答を記入。ここに解答を記入。ここに解答を記入。ここに解答を記入。ここに解答を記入。ここに解答を記入。ここに解答を記入。ここに解答を記入。ここに解答を記入。ここに解答を記入。ここに解答を記入。ここに解答を記入。ここに解答を記入。

\columnbreak %右ページに移動

\noindent ここに解説を記入。ここに解説を記入。ここに解説を記入。ここに解説を記入。ここに解説を記入。ここに解説を記入。ここに解説を記入。ここに解説を記入。ここに解説を記入。ここに解説を記入。ここに解説を記入。ここに解説を記入。ここに解説を記入。ここに解説を記入。ここに解説を記入。ここに解説を記入。ここに解説を記入。ここに解説を記入。ここに解説を記入。ここに解説を記入。ここに解説を記入。ここに解説を記入。ここに解説を記入。ここに解説を記入。ここに解説を記入。ここに解説を記入。ここに解説を記入。ここに解説を記入。ここに解説を記入。ここに解説を記入。ここに解説を記入。ここに解説を記入。ここに解説を記入。ここに解説を記入。ここに解説を記入。ここに解説を記入。ここに解説を記入。ここに解説を記入。ここに解説を記入。ここに解説を記入。
\end{multicols}

\begin{itembox}[l]{{\bf 検討} ここにタイトル}
ここに関連知識を記入。ここに関連知識を記入。ここに関連知識を記入。ここに関連知識を記入。ここに関連知識を記入。ここに関連知識を記入。ここに関連知識を記入。ここに関連知識を記入。ここに関連知識を記入。ここに関連知識を記入。ここに関連知識を記入。ここに関連知識を記入。ここに関連知識を記入。ここに関連知識を記入。ここに関連知識を記入。ここに関連知識を記入。ここに関連知識を記入。ここに関連知識を記入。ここに関連知識を記入。ここに関連知識を記入。
\end{itembox}

\end{kaitou}

\rennsyuu

ここに練習問題を記入。ここに練習問題を記入。ここに練習問題を記入。ここに練習問題を記入。ここに練習問題を記入。ここに練習問題を記入。ここに練習問題を記入。ここに練習問題を記入。ここに練習問題を記入。ここに練習問題を記入。ここに練習問題を記入。ここに練習問題を記入。ここに練習問題を記入。ここに練習問題を記入。ここに練習問題を記入。ここに練習問題を記入。ここに練習問題を記入。ここに練習問題を記入。ここに練習問題を記入。ここに練習問題を記入。
\end{document}
