\documentclass[12pt]{jarticle}
\usepackage{fancyhdr}
\usepackage{amsmath}
\usepackage{multicol}
\usepackage{tcolorbox}
\usepackage{ascmac}
\usepackage{comment}
\usepackage{enumerate}
\usepackage{fancybox}
\usepackage{framed}
\usepackage{setspace}
\usepackage{tikz}
\usepackage{enumitem}
\usepackage{tasks}
\usepackage{ifthen}
\usepackage[top=10mm, bottom=28mm, left=17mm,right=17mm]{geometry} %%%%%%%%%%%%%%設定色々%%%%%%%%%%%%%%%
\pagestyle{fancy}
\cfoot{}
\renewcommand{\headrulewidth}{0pt}
\setlength{\columnseprule}{0.5pt}
\setlength{\columnwidth}{5mm}
\newcommand{\tanngenn}[1]{\lhead{\framebox[176mm][l]{#1}}}
\newenvironment{reidai}[1]{\noindent{\Large\colorbox[gray]{0.8}{基}\fcolorbox{white}{gray}{\textcolor{white}{本}}\colorbox[gray]{0.8}{例}\fcolorbox{white}{gray}{\textcolor{white}{題}}} {\large \hspace{2mm} #1 \hspace{2mm}}\hrulefill\\ \par }{\par \noindent\hrulefill\\}
\newcommand{\sisinn}[1]{\noindent{\bf 指針} #1}
\newcommand{\rennsyuu}{\noindent\fcolorbox{white}{gray}{\textcolor{white}{練}}\colorbox[gray]{0.8}{習}\fcolorbox{white}{gray}{\textcolor{white}{問}}\colorbox[gray]{0.8}{題}}
\newenvironment{ouyou}[1]{\noindent{\Large\colorbox[gray]{0.8}{応}\fcolorbox{white}{gray}{\textcolor{white}{用}}\colorbox[gray]{0.8}{例}\fcolorbox{white}{gray}{\textcolor{white}{題}}} {\large \hspace{2mm} #1 \hspace{2mm}}\hrulefill\\ \par }{\par \noindent\hrulefill\\}
\newenvironment{kaitou}{\noindent{\bf 解答} \dotfill}{\noindent\\}
%%%%%%%%%%%%%%ここから本文開始%%%%%%%%%%%%%%%
\begin{document}
\begin{reidai}{ド・モアブルの定理}
\noindent $\displaystyle \left(\cos\frac{\pi}{3} + i \sin\frac{\pi}{3}\right)^4$や$\displaystyle \left(\cos\frac{\pi}{6} + i \sin\frac{\pi}{6}\right)^{-6}$を求めなさい。
\end{reidai}
\sisinn{ド・モアブルの定理}
\begin{itemize}
  \item[]
  \item[]
  \item[]
\end{itemize}
\begin{kaitou}
\begin{multicols}{2}
\noindent
\begin{itemize}
  \item[]
  \item[]
  \item[]
  \item[]
  \item[]
  \item[]
\end{itemize}
\columnbreak %右ページに移動
\begin{itemize}
  \item[]
  \item[]
  \item[]
  \item[]
  \item[]
  \item[]
\end{itemize}
\noindent
\end{multicols}
\end{kaitou}
\begin{reidai}{複素数の$n$乗根}
$1$の$6$乗根を表す複素数を求めなさい。
\end{reidai}
\sisinn{$1$の$n$乗根}
\begin{itemize}
  \item[]
  \item[]
  \item[]
\end{itemize}
\begin{kaitou}
  \begin{multicols}{2}
  \noindent
  \begin{itemize}
    \item[]
    \item[]
    \item[]
    \item[]
    \item[]
  \end{itemize}
  \columnbreak %右ページに移動
  \begin{itemize}
    \item[]
    \item[]
    \item[]
    \item[]
    \item[]
    \item[]
  \end{itemize}
  \noindent
  \end{multicols}
\end{kaitou}
\begin{ouyou}{}
方程式$z^3 = 8i$を解きなさい。
\end{ouyou}
\sisinn{}
\begin{itemize}
  \item[]
  \item[]
  \item[]
\end{itemize}
\begin{kaitou}
  \begin{multicols}{2}
  \noindent
  \begin{itemize}
    \item[]
    \item[]
    \item[]
    \item[]
    \item[]
  \end{itemize}
  \columnbreak %右ページに移動
  \begin{itemize}
    \item[]
    \item[]
    \item[]
    \item[]
    \item[]
    \item[]
  \end{itemize}
  \noindent
  \end{multicols}
\end{kaitou}

\rennsyuu
\subsection*{次の式を計算しなさい。}
\begin{itemize}
  \item[(1)] $\displaystyle \left(\sqrt{3} + i \right)^6$
  \item[(2)] $\displaystyle \left(1 + \sqrt{3}i \right)^5$
  \item[(3)] $\displaystyle \left(1 -  i \right)^8$
  \item[(4)] $\displaystyle \left(1 - \sqrt{3}i \right)^{-6}$
\end{itemize}
\subsection*{$1$の$8$乗根を求めなさい。}
\end{document}
