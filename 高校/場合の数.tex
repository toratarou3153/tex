\documentclass[twocolumn, 20pt]{jarticle}
\usepackage{a4wide}
\usepackage{amsmath}%数学記号
\usepackage{amssymb}%数学記号
\usepackage{epsfig}%図
\usepackage{latexsym}
\usepackage{supertabular}
\usepackage{graphicx}
\usepackage{color}
\usepackage{ascmac}
\usepackage{ascmac}
\usepackage{systeme}
\usepackage{amsmath,cases}
\pagestyle{plain}

\newtheorem{theorem}{定理}[section]
\newtheorem{lemma}[theorem]{補題}
\newtheorem{proposition}[theorem]{命題}
\newtheorem{conjecture}[theorem]{予想}
\newtheorem{corollary}[theorem]{系}
\newtheorem{definition}[theorem]{定義}
\newtheorem{example}[theorem]{例}
\newtheorem{exercise}[theorem]{例題}
\newtheorem{problem}[theorem]{問}
\newtheorem{algorithm}[theorem]{アルゴリズム}
\newtheorem{remark}[theorem]{注意}

\def\qed{{\hfill$\square$}}
\def\proof{{\vspace{-0.3cm}f 証明: \,}}
\def\solution{{\vspace{-0.3cm}f 解: \,}}
\def\N{{\Bbb N}}
\def\Z{{\Bbb Z}}
\def\Q{{\Bbb Q}}
\def\R{{\Bbb R}}
\def\C{{\Bbb C}}
\def\F{{\Bbb F}}
\def\D{{\mathcal D}}
\def\mod{{\mathrm{mod\,\,}}}
\def\GL{{\mathrm{GL}}}
\def\GF{{\mathrm{GF}}}
\def\H{{\mathcal{H}}}

\setlength{\textwidth}{170mm}
\setlength{\textheight}{240mm}
\setlength{\oddsidemargin}{-5mm}
\setlength{\evensidemargin}{-5mm}
\setlength{\topmargin}{-10mm}
\setlength{\headheight}{0mm}
\setlength{\headsep}{10mm}
\setlength{\columnsep}{5mm}

\title{場合の数、組み合わせ}
\begin{document}
\columnseprule=0.2mm
\date{}
\maketitle
\section{基本問題6選}
\begin{itemize}
  \item 15人を8人と7人の2組に分ける
\end{itemize}
---------------------------------------------------------------
\\
\\
\\
\\
\\
\\
\\
\\
\\
---------------------------------------------------------------
\begin{itemize}
  \item 15人を2組に分ける
\end{itemize}
---------------------------------------------------------------
\\
\\
\\
\\
\\
\\
\\
\\
\\
---------------------------------------------------------------
\begin{itemize}
  \item 15人を6人と5人と4人の3組に分ける
\end{itemize}
---------------------------------------------------------------
\\
\\
\\
\\
\\
\\
\\
\\
\\
\\
\\
---------------------------------------------------------------

\begin{itemize}
  \item 15人を7人と4人と4人の3組に分ける
\end{itemize}
---------------------------------------------------------------
\\
\\
\\
\\
\\
\\
\\
\\
\\
---------------------------------------------------------------
\begin{itemize}
  \item 15人を5人ずつ3組に分ける
\end{itemize}
---------------------------------------------------------------
\\
\\
\\
\\
\\
\\
\\
\\
\\
---------------------------------------------------------------
\begin{itemize}
  \item 15人をA,B,C組に分ける(0人も可)
\end{itemize}
---------------------------------------------------------------
\\
\\
\\
\\
\\
\\
\\
\\
\\
---------------------------------------------------------------

右図の直角三角形において、\\
\qquad \qquad $\displaystyle \sin \theta = $\\であるから、次が成り立つ。\\
\qquad \qquad $\displaystyle y= $

\end{document}
