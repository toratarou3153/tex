\documentclass[12pt]{jarticle}
\usepackage{a4wide}
\usepackage{amsmath}%数学記号
\usepackage{amssymb}%数学記号
\usepackage{epsfig}%図
\usepackage{latexsym}
\usepackage{supertabular}
\usepackage{graphicx}
\usepackage{color}
\usepackage{ascmac}
\usepackage{multicol}
\usepackage{ascmac}
\usepackage{systeme}
\usepackage{amsmath,cases}
\pagestyle{plain}

\newtheorem{theorem}{定理}[section]
\newtheorem{lemma}[theorem]{補題}
\newtheorem{proposition}[theorem]{命題}
\newtheorem{conjecture}[theorem]{予想}
\newtheorem{corollary}[theorem]{系}
\newtheorem{definition}[theorem]{定義}
\newtheorem{example}[theorem]{例}
\newtheorem{exercise}[theorem]{例題}
\newtheorem{problem}[theorem]{問}
\newtheorem{algorithm}[theorem]{アルゴリズム}
\newtheorem{remark}[theorem]{注意}

\def\qed{{\hfill$\square$}}
\def\proof{{\vspace{-0.3cm}f 証明: \,}}
\def\solution{{\vspace{-0.3cm}f 解: \,}}
\def\N{{\Bbb N}}
\def\Z{{\Bbb Z}}
\def\Q{{\Bbb Q}}
\def\R{{\Bbb R}}
\def\C{{\Bbb C}}
\def\F{{\Bbb F}}
\def\D{{\mathcal D}}
\def\mod{{\mathrm{mod\,\,}}}
\def\GL{{\mathrm{GL}}}
\def\GF{{\mathrm{GF}}}
\def\H{{\mathcal{H}}}

\setlength{\textwidth}{170mm}
\setlength{\textheight}{240mm}
\setlength{\oddsidemargin}{-5mm}
\setlength{\evensidemargin}{-5mm}
\setlength{\topmargin}{-45mm}
\setlength{\headheight}{0mm}
\setlength{\headsep}{10mm}

\title{確認プリント$\cdot$ベクトル}
\begin{document}
\date{}
\maketitle
\vspace{-3cm}
\subsection*{$2$つのベクトル$\displaystyle \overrightarrow{a}$,$\displaystyle \overrightarrow{b}$について次の問いに答えなさい。}
\begin{itemize}
  \item[(1)] $2$つのベクトルの内積の計算式を書きなさい。
  \item[(2)] $2$つのベクトルの垂直条件を書きなさい。
  \item[(3)] $2$つのベクトルのなす角$\theta$について$\cos{\theta}$を求める式を書きなさい。
  \item[(4)] $2$つのベクトルが作る三角形の面積を求める式を書きなさい。
  \item[(5)] $3$点$A, B, C$が作る$\triangle{ABC}$の面積を求める式を書きなさい。
\end{itemize}
(回答欄)\\
\\
\\
\\
\subsection*{座標空間における次の問いに答えなさい。}
\begin{itemize}
  \item[(1)] $3$点$A, B, C$が一直線上にある条件を書きなさい。
  \item[(2)] $3$点$A, B, C$が同じ平面上にある条件を$2$パターン書きなさい。
\end{itemize}
(回答欄)\\
\\
\\
\\
\\
\\
\\
\subsection*{四面体OABCにおいてOAの中点をM、BCを$2:1$に内分する点をNとする。このとき、以下の問いに答えなさい。}
\begin{itemize}
  \item[(1)] $\displaystyle \overrightarrow{OM}$を$\displaystyle \overrightarrow{OA}$, $\displaystyle \overrightarrow{OB}$, $\displaystyle \overrightarrow{OC}$を用いて表しなさい。
  \item[(2)]  $\displaystyle \overrightarrow{ON}$を$\displaystyle \overrightarrow{OA}$, $\displaystyle \overrightarrow{OB}$, $\displaystyle \overrightarrow{OC}$を用いて表しなさい。
  \item[(3)] $\displaystyle \overrightarrow{MN}$を$\displaystyle \overrightarrow{OA}$, $\displaystyle \overrightarrow{OB}$, $\displaystyle \overrightarrow{OC}$を用いて表しなさい。
\end{itemize}
(回答欄)
\thispagestyle{empty}
\end{document}
