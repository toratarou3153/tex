\documentclass[12pt]{jarticle}
\usepackage{a4wide}
\usepackage{amsmath}%数学記号
\usepackage{amssymb}%数学記号
\usepackage{epsfig}%図
\usepackage{latexsym}
\usepackage{supertabular}
\usepackage{graphicx}
\usepackage{color}
\usepackage{ascmac}
\usepackage{multicol}
\usepackage{ascmac}
\usepackage{systeme}
\usepackage{amsmath,cases}
\pagestyle{plain}

\newtheorem{theorem}{定理}[section]
\newtheorem{lemma}[theorem]{補題}
\newtheorem{proposition}[theorem]{命題}
\newtheorem{conjecture}[theorem]{予想}
\newtheorem{corollary}[theorem]{系}
\newtheorem{definition}[theorem]{定義}
\newtheorem{example}[theorem]{例}
\newtheorem{exercise}[theorem]{例題}
\newtheorem{problem}[theorem]{問}
\newtheorem{algorithm}[theorem]{アルゴリズム}
\newtheorem{remark}[theorem]{注意}

\def\qed{{\hfill$\square$}}
\def\proof{{\vspace{-0.3cm}f 証明: \,}}
\def\solution{{\vspace{-0.3cm}f 解: \,}}
\def\N{{\Bbb N}}
\def\Z{{\Bbb Z}}
\def\Q{{\Bbb Q}}
\def\R{{\Bbb R}}
\def\C{{\Bbb C}}
\def\F{{\Bbb F}}
\def\D{{\mathcal D}}
\def\mod{{\mathrm{mod\,\,}}}
\def\GL{{\mathrm{GL}}}
\def\GF{{\mathrm{GF}}}
\def\H{{\mathcal{H}}}

\setlength{\textwidth}{170mm}
\setlength{\textheight}{240mm}
\setlength{\oddsidemargin}{-5mm}
\setlength{\evensidemargin}{-5mm}
\setlength{\topmargin}{-10mm}
\setlength{\headheight}{0mm}
\setlength{\headsep}{10mm}


\title{模擬中間テスト}
\begin{document}
\date{}
\maketitle
\vspace{-2cm}
\subsection{次の関数を微分しなさい}
\begin{itemize}
  \item [(1)] $ \displaystyle y = 4x^2 + 3x -4$
  \item [(2)] $\displaystyle y = 2x^2 -5x + 1$
  \item [(3)] $\displaystyle y = -3x^2 + x -2$
  \item [(4)] $\displaystyle y = -x^2 -x + 3$
  \item [(5)] $\displaystyle y = x^2 + 2x^2 -3x$
  \item [(6)] $\displaystyle y = -2x^2 -x^2 + 6x -2$
  \item [(7)] $\displaystyle \frac{4}{3}x^3 + \frac{3}{4}x^2-\frac{1}{2}x$
  \item [(8)] $\displaystyle -\frac{1}{3}x^3 + \frac{3}{2}x^2-\frac{1}{2}$
\end{itemize}

\subsection{$f(x)=x^3-3x^2+3$について次のxの値における微分係数を求めなさい}
$(1)x=2$

$(2)x=0$

$(3)x=-2$
\subsection{関数$y=2x^2 -4x +3$のグラフ上に点$(1, 3)$をとる。}
$(1)$点Aにおける接線の傾きを求めなさい。

$(2)$点Aにおける接線の方程式を求めなさい。

\subsection{関数$y=x^2+3$のグラフ上に点C$(1, 0)$}




\end{document}
