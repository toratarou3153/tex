\documentclass[12pt]{jarticle}
\usepackage{a4wide}
\usepackage{amsmath}%数学記号
\usepackage{amssymb}%数学記号
\usepackage{epsfig}%図
\usepackage{latexsym}
\usepackage{supertabular}
\usepackage{graphicx}
\usepackage{color}
\usepackage{ascmac}
\usepackage{multicol}
\usepackage{ascmac}
\usepackage{systeme}
\usepackage{amsmath,cases}
\pagestyle{plain}

\newtheorem{theorem}{定理}[section]
\newtheorem{lemma}[theorem]{補題}
\newtheorem{proposition}[theorem]{命題}
\newtheorem{conjecture}[theorem]{予想}
\newtheorem{corollary}[theorem]{系}
\newtheorem{definition}[theorem]{定義}
\newtheorem{example}[theorem]{例}
\newtheorem{exercise}[theorem]{例題}
\newtheorem{problem}[theorem]{問}
\newtheorem{algorithm}[theorem]{アルゴリズム}
\newtheorem{remark}[theorem]{注意}

\def\qed{{\hfill$\square$}}
\def\proof{{\vspace{-0.3cm}f 証明: \,}}
\def\solution{{\vspace{-0.3cm}f 解: \,}}
\def\N{{\Bbb N}}
\def\Z{{\Bbb Z}}
\def\Q{{\Bbb Q}}
\def\R{{\Bbb R}}
\def\C{{\Bbb C}}
\def\F{{\Bbb F}}
\def\D{{\mathcal D}}
\def\mod{{\mathrm{mod\,\,}}}
\def\GL{{\mathrm{GL}}}
\def\GF{{\mathrm{GF}}}
\def\H{{\mathcal{H}}}

\setlength{\textwidth}{170mm}
\setlength{\textheight}{240mm}
\setlength{\oddsidemargin}{-5mm}
\setlength{\evensidemargin}{-5mm}
\setlength{\topmargin}{-10mm}
\setlength{\headheight}{0mm}
\setlength{\headsep}{10mm}

\title{複素数と方程式}
\begin{document}
\maketitle
\section*{練習問題}
次の計算をせよ。
\begin{itemize}
  \item[(1)]  $\displaystyle (2+3i)+(4+i)$
  \item[(2)] $\displaystyle (-1+2i) + 3-4i$
  \item[(3)] $\displaystyle (6+4i)- (3+2i)$
  \item[(4)] $\displaystyle (2-3i)-(4-2i)$
  \item[(5)] $\displaystyle (1+2i)(4+3i)$
  \item[(6)] $\displaystyle (2-i)(3+4i)$
  \item[(7)] $\displaystyle (3+4i)(3-4i)$
  \item[(8)] $\displaystyle (2+3i)^2$
\end{itemize}
次の複素数と共役な複素数を求めよ。
\begin{itemize}
  \item [(1)]$\displaystyle 2+3i$
  \item [(2)]$\displaystyle 1-i$
  \item [(3)]$\displaystyle \sqrt{3}i$
  \item [(4)]$\displaystyle \frac{-1+\sqrt{3}i}{2}$
\end{itemize}
次の計算をせよ。
\begin{itemize}
  \item [(1)]$\displaystyle \frac{1+2i}{2+3i}$
  \item [(2)]$\displaystyle \frac{1-i}{1+i}$
  \item [(3)]$\displaystyle \frac{5i}{2-i}$
  \item [(4)]$\displaystyle \frac{1}{i}$
  \item [(5)]$\displaystyle \sqrt{-2}\sqrt{-6}$
  \item [(6)]$\displaystyle \sqrt{-6}\sqrt{3}$
  \item [(7)]$\displaystyle \frac{\sqrt{-8}}{\sqrt{2}}$
  \item [(8)]$\displaystyle \frac{\sqrt{-3}}{\sqrt{-2}}$
\end{itemize}
授業を受けての感想、質問:

\end{document}
