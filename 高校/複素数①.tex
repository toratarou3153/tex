\documentclass[12pt]{jarticle}
\usepackage{a4wide}
\usepackage{amsmath}%数学記号
\usepackage{amssymb}%数学記号
\usepackage{epsfig}%図
\usepackage{latexsym}
\usepackage{supertabular}
\usepackage{graphicx}
\usepackage{color}
\usepackage{ascmac}
\usepackage{multicol}
\usepackage{ascmac}
\usepackage{systeme}
\usepackage{amsmath,cases}
\pagestyle{plain}

\newtheorem{theorem}{定理}[section]
\newtheorem{lemma}[theorem]{補題}
\newtheorem{proposition}[theorem]{命題}
\newtheorem{conjecture}[theorem]{予想}
\newtheorem{corollary}[theorem]{系}
\newtheorem{definition}[theorem]{定義}
\newtheorem{example}[theorem]{例}
\newtheorem{exercise}[theorem]{例題}
\newtheorem{problem}[theorem]{問}
\newtheorem{algorithm}[theorem]{アルゴリズム}
\newtheorem{remark}[theorem]{注意}

\def\qed{{\hfill$\square$}}
\def\proof{{\vspace{-0.3cm}f 証明: \,}}
\def\solution{{\vspace{-0.3cm}f 解: \,}}
\def\N{{\Bbb N}}
\def\Z{{\Bbb Z}}
\def\Q{{\Bbb Q}}
\def\R{{\Bbb R}}
\def\C{{\Bbb C}}
\def\F{{\Bbb F}}
\def\D{{\mathcal D}}
\def\mod{{\mathrm{mod\,\,}}}
\def\GL{{\mathrm{GL}}}
\def\GF{{\mathrm{GF}}}
\def\H{{\mathcal{H}}}

\setlength{\textwidth}{170mm}
\setlength{\textheight}{240mm}
\setlength{\oddsidemargin}{-5mm}
\setlength{\evensidemargin}{-5mm}
\setlength{\topmargin}{-10mm}
\setlength{\headheight}{0mm}
\setlength{\headsep}{10mm}

\title{複素数と方程式}
\date{$2022年4月12日$}
\begin{document}
\maketitle
\author{名前:}
\section{複素数と方程式}
次のような実数$x,y$を求めなさい。
\begin{itemize}
  \item[(1)]  $\displaystyle (x -3) + (x + y)i = 0$\\
  \\
  $x = 3,y=-3$
  \\
  \item[(2)] $(3x + y) + (y +3)i = 0$
  \\
  \\
  $x = 1,y=-3$
  \\
  \item[(3)] $(x + 2y) + 2yi = 5 + 4i$
  \\
  \\
  $x = 1,y=2$
  \\
  \item[(4)] $\displaystyle (x - 2y) + (2x - 3y)i = 4 + 7i$
  \\
  \\
  $x = 2,y=-1$
  \\
  \item[(5)] $\displaystyle (2 + i)x+ (3 - 2i)y = -9 + 20i$
  \\
  \\
  $x = 6,y=-7$
  \\
  \item[(6)] $\displaystyle (4 + 3i)x + (1 + 2i)y + 5 = 0$
  \\
  \\
  $x = -2,y=3$
\end{itemize}

\newpage
\ \ \ \ \ \ \ \ \ \ \ \ \ \ $\displaystyle (4 + 3i)x + (1 + 2i)y + 5 = 0$の解説:

\ \ \ \ \ \ \ \ \ \ \ \ \ \ 展開して$i$についてまとめなおすと
\begin{align*}
  \displaystyle
  & 4x + 3ix +y +2iy +5 = 0&\\
  & (4x + y + 5) + (3x + 2y)i = 0&
\end{align*}
\ \ \ \ \ \ \ \ \ \ \ \ \ \ \ $x,y$は実数なので$(4x + y + 5)$と$(3x + 2y)$も実数となる。

\ \ \ \ \ \ \ \ \ \ \ \ \ \ 右辺が$0$なので、$(4x + y + 5)= 0$、$(3x + 2y) = 0$となる。

\ \ \ \ \ \ \ \ \ \ \ \ \ \ これを解くと$x = -2,y = 3$となる。
\\
\\
\begin{align}
  \displaystyle
  3x^2 -7x + 5 &=0\\
  判別式D &= (-7)^2 - 4 \cdot 3 \cdot 5\\
  &= -11 < 0\\
  よって、異なる&2つの虚数解をもつ。
\end{align}
\\
\\


\end{document}
